{\actuality} 
Жизнь  человека с незапамятных времён во многом зависит от состояния атмосферы, которая является нашей естественной средой обитания. Погода оказывает существенное влияние на многие аспекты хозяйственной деятельности. Её колебания крайне важны для таких отраслей, как сельское хозяйство, рыбная ловля, авиаперевозки, морской и речной транспорт, энергетика и многие другие. Опасные погодные явления не редко становятся причиной масштабных бедствий, которые поражают значительные территории различных государств по всему миру (например, ураганы <<Катрина>> в США в~2005 году и <<Айк>> на Гаити в~2008 году, аномальная жара с лесными пожарами в Росcии и странах Европы в~2010 году, разрушительные наводнения в Краснодарском крае в~2012 году и на Дальнем Востоке в~2013 году и~мн.\:др.). Технический прогресс заметно повысил способность человека противостоять таким погодным явлениям, в т.\:ч. путём прогноза их появления и развития.

Вот почему сегодняшний мир невозможно представить без своевременного и точного прогноза погоды. Ошибки в предсказании грядущих погодных условий могут помешать грамотной и адекватной подготовке к возникающим опасным метеорологическим явлениям. Тогда как вовремя сделанное предупреждение может существенно уменьшить, а в некоторых случаях и совершенно исключить, материальный ущерб и людские потери. Этот факт является одной из причин развития методов прогноза погоды, безусловно являющегося исключительно важной задачей, для решения которой в развитых странах созданы и успешно работают целые научно-исследовательские центры.

Бурный рост вычислительных мощностей, берущий своё начало во второй половине XX века и стремительно продолжающийся в XXI веке, стал одной из причин развития \textbf{численного прогноза погоды} как важного направления научной и практической деятельности мировых прогностических центров.

На данном этапе развития технологий любая система численного прогноза погоды представляет собой сложный программно-вычислительный комплекс, содержащий помимо численной математической модели, которая воспроизводит динамику атмосферной циркуляции, также блок, ответственный за подготовку начальных данных. Подготовка полей с начальными данными, необходимых для старта численной модели прогноза погоды, в метеорологии происходит с помощью т.\:н. \textbf{усвоения данных}, которое занимает центральное место в данной работе.

В метеорологии усвоением данных называют циклический процесс, в ходе которого определяются численные характеристики состояния атмосферы в определённые моменты времени с использованием имеющихся данных метеорологических наблюдений и некоторой априорной информации об исследуемом состоянии атмосферы. B результате получают поля так называемого объективного анализа, определённого в точках некоторой предварительно заданной сетки.
Главной целью проведения процедуры усвоения данных, как было отмечено выше, является использование полученных полей объективного анализа в качестве начальных данных при интегрировании численных моделей прогноза погоды.
Естественно, что в этих условиях объективный анализ состоит из значений модельных переменных, определённых в точках модельной сетки. Размерности этих полей велики и продолжают постоянно расти вследствие непрекращающегося роста разрешающих способностей модельных сеток. Так, используемая в данной работе версия глобальной модели ПЛАВ~\cite{Tolstykh-SLAV}, которая является оперативной в Гидрометцентре России, имеет постоянное разрешение по пространству ${0.9^{\circ}\times0,72^{\circ}}$, что составляет 400 узлов сетки по долготе и 251 узел по широте. Переменные в свободной атмосфере (две компоненты ветра, температура и влажность воздуха) расположены на 28 вертикальных уровнях. Кроме того, в модели используются 5 переменных (те же, что и в атмосфере, а также давление) на поверхности Земли, которые тоже должны быть инициированы начальными значениями. Таким образом, общая размерность поля объективного анализа составляет величину порядка ${10^7}$.

По мнению Л.С. Гандина, который является автором одного из наиболее долго и успешно применяемого метода, называемого <<Оптимальная интерполяция>>, термин <<объективный анализ>> не является удачным, т.к. в ходе его получения происходит не исследование свойств полей метеорологических элементов (собственно, анализ), а, скорее, восстановление значений этих полей по данным метеорологических наблюдений~\cite[с.~5]{Gandin-1963}. Тем не менее, этот термин получил широкое применение уже тогда и в дальнейшем ещё более укрепился в своём использовании.

Объективный анализ, чаще называемый просто \textbf{<<анализ>>}, используют не только для инициализации численных моделей прогноза погоды. Поля анализа применяются также для оценки успешности численных прогнозов. При этом анализ выступает в качестве наилучшего приближения к истинному состоянию атмосферы. Комиссия по общим системам Всемирной метеорологической организации (CBS/WMO) определила перечень стандартных процедур~\cite{WMO-CBS}, с помощью которых производится расчёт оценок ошибок прогнозов. Эти оценки рассчитываются всеми прогностическими центрами. Ежемесячно происходит обмен информацией об ошибках прогнозов. Эта информация доступна на сайте Ведущего центра по верификации детерминированных численных прогнозов (Lead Centre for Deterministic Forecast Verification LCDNV, http://apps.ecmwf.int/wmolcdnv/). В данной работе указанные процедуры также применяются для оценки качества прогнозов. Более подробное описание используемых оценок будет приведено в главе~\ref{chapt4}.

Часто поля анализа применяются также для получения псевдонаблюдений. На основе конкретных значений анализов формируют выборки наблюдений, обладающих заданными известными свойствами (в первую очередь с известной функцией распределения и её параметрами). В дальнейшем полученные таким образом <<наблюдения>> используются в различных исследованиях.

Аналогично получают и используют поля анализов в области моделирования и исследования мирового океана, а также в получивших развитие последнее время совместных моделях атмосферы, океана и морского льда.

С того момента как для формирования таких начальных данных в 1963 году \cite{Gandin-1963} была использована техника статистической интерполяции, стартовые поля численных моделей прогноза погоды получают путём некоторой оптимальной оценки состояния атмосферы, выполненной на основе взвешенной суммы так называемого первого приближения и метеорологических наблюдений.


Число метеорологических наблюдений хотя и постоянно растёт, но всё же по-прежнему меньше количества узлов модельной сетки. Такая оценка называется объективным анализом (или чаще просто анализом), а процесс по его получению называется усвоением данных.

\newpage
Обзор, введение в тему, обозначение места данной работы в
мировых исследованиях и~т.\:п., можно использовать ссылки на другие
работы~ (если их нет, то в автореферате
автоматически пропадёт раздел <<Список литературы>>).

Внимание! Ссылки
на другие работы в разделе общей характеристики работы можно
использовать только при использовании \verb!biblatex! (из-за технических
ограничений \verb!bibtex8!. Это связано с тем, что одна и та же
характеристика используются и в тексте диссертации, и в
автореферате. В последнем, согласно ГОСТ, должен присутствовать список
работ автора по теме диссертации, а \verb!bibtex8! не умеет выводить в одном
файле два списка литературы).

 \aim\ данной работы является \ldots

Для~достижения поставленной цели необходимо было решить следующие {\tasks}:
\begin{enumerate}
  \item Исследовать, разработать, вычислить и~т.\:д. и~т.\:п.
  \item Исследовать, разработать, вычислить и~т.\:д. и~т.\:п.
  \item Исследовать, разработать, вычислить и~т.\:д. и~т.\:п.
  \item Исследовать, разработать, вычислить и~т.\:д. и~т.\:п.
\end{enumerate}

\defpositions
\begin{enumerate}
  \item Первое положение
  \item Второе положение
  \item Третье положение
  \item Четвертое положение
\end{enumerate}

\novelty
\begin{enumerate}
  \item Впервые \ldots
  \item Впервые \ldots
  \item Было выполнено оригинальное исследование \ldots
\end{enumerate}

\influence\ \ldots

\reliability\ полученных результатов обеспечивается \ldots \ Результаты находятся в соответствии с результатами, полученными другими авторами.

\probation\
Основные результаты работы докладывались~на:
перечисление основных конференций, симпозиумов и~т.\:п.

\contribution\ Автор принимал активное участие \ldots

%\publications\ Основные результаты по теме диссертации изложены в ХХ печатных изданиях~\cite{Sokolov,Gaidaenko,Lermontov,Management},
%Х из которых изданы в журналах, рекомендованных ВАК~\cite{Sokolov,Gaidaenko}, 
%ХХ --- в тезисах докладов~\cite{Lermontov,Management}.

\ifthenelse{\equal{\thebibliosel}{0}}{% Встроенная реализация с загрузкой файла через движок bibtex8
    \publications\ Основные результаты по теме диссертации изложены в XX печатных изданиях, 
    X из которых изданы в журналах, рекомендованных ВАК, 
    X "--- в тезисах докладов.%
}{% Реализация пакетом biblatex через движок biber
%Сделана отдельная секция, чтобы не отображались в списке цитированных материалов
    \begin{refsection}%
        \printbibliography[heading=countauthornotvak, env=countauthornotvak, keyword=biblioauthornotvak, section=1]%
        \printbibliography[heading=countauthorvak, env=countauthorvak, keyword=biblioauthorvak, section=1]%
        \printbibliography[heading=countauthorconf, env=countauthorconf, keyword=biblioauthorconf, section=1]%
        \printbibliography[heading=countauthor, env=countauthor, keyword=biblioauthor, section=1]%
        \publications\ Основные результаты по теме диссертации изложены в \arabic{citeauthor} печатных изданиях\nocite{bib1,bib2,vak-Shlyaeva-2013,vak-Mizyak-2013}, 
        \arabic{citeauthorvak} из которых изданы в журналах, рекомендованных ВАК\nocite{vak-Shlyaeva-2013,vak-Mizyak-2013}, 
        \arabic{citeauthorconf} "--- в тезисах докладов\nocite{confbib1,confbib2}.
    \end{refsection}
}
При использовании пакета \verb!biblatex! для автоматического подсчёта
количества публикаций автора по теме диссертации, необходимо
их здесь перечислить с использованием команды \verb!\nocite!.
    


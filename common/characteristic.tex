{\actuality} Сегодняшний мир невозможно представить без своевременного и точного прогноза погоды. Ошибки в предсказании грядущих погодных условий могут помешать грамотной и адекватной подготовке к возникающим опасным метеорологическим явлениям. Тогда как вовремя сделанное предупреждение может существенно уменьшить, а в некоторых случаях и совершенно исключить, материальный ущерб и людские потери. В том числе и по этой причине развитие методов прогноза погоды сегодня является исключительно важной задачей, для решения которой в развитых странах созданы и успешно работают целые научно-исследовательские центры.


Бурный рост вычислительных мощностей, берущий своё начало во второй половине XX веке и стремительно продолжающийся в XXI веке, стал одной из причин развития численного прогноза погоды как важного направления научной и практической деятельности мировых прогностических центров.

На данном этапе развития технологий любая система численного прогноза погоды представляют собой сложный программно-вычислительный комплекс, содержащий помимо численной математической модели, которая воспроизводит динамику атмосферной циркуляции, также блок, ответственный за подготовку начальных данных. С того момента как для формирования таких начальных данных в 1963 году \cite{Gandin-1963} была использована техника статистической интерполяции, стартовые поля численных моделей прогноза погоды получают путём некоторой оптимальной оценки состояния атмосферы, выполненной на основе взвешенной суммы так называемого первого приближения и метеорологических наблюдений.


Число метеорологических наблюдений хотя и постоянно растёт, но всё же по-прежнему меньше количества узлов модельной сетки. Такая оценка называется объективным анализом (или чаще просто анализом), а процесс по его получению называется усвоением данных.

Обзор, введение в тему, обозначение места данной работы в
мировых исследованиях и~т.\:п., можно использовать ссылки на другие
работы~ (если их нет, то в автореферате
автоматически пропадёт раздел <<Список литературы>>).
Тут~я пытаюсь использовать ссылку на свою статью \cite{Mizyak-2013}. Ещё мои работы --- \cite{Slyaeva-2013} и \cite{Tolstykh-2015} 

Внимание! Ссылки
на другие работы в разделе общей характеристики работы можно
использовать только при использовании \verb!biblatex! (из-за технических
ограничений \verb!bibtex8!. Это связано с тем, что одна и та же
характеристика используются и в тексте диссертации, и в
автореферате. В последнем, согласно ГОСТ, должен присутствовать список
работ автора по теме диссертации, а \verb!bibtex8! не умеет выводить в одном
файле два списка литературы).

 \aim\ данной работы является \ldots

Для~достижения поставленной цели необходимо было решить следующие {\tasks}:
\begin{enumerate}
  \item Исследовать, разработать, вычислить и~т.\:д. и~т.\:п.
  \item Исследовать, разработать, вычислить и~т.\:д. и~т.\:п.
  \item Исследовать, разработать, вычислить и~т.\:д. и~т.\:п.
  \item Исследовать, разработать, вычислить и~т.\:д. и~т.\:п.
\end{enumerate}

\defpositions
\begin{enumerate}
  \item Первое положение
  \item Второе положение
  \item Третье положение
  \item Четвертое положение
\end{enumerate}

\novelty
\begin{enumerate}
  \item Впервые \ldots
  \item Впервые \ldots
  \item Было выполнено оригинальное исследование \ldots
\end{enumerate}

\influence\ \ldots

\reliability\ полученных результатов обеспечивается \ldots \ Результаты находятся в соответствии с результатами, полученными другими авторами.

\probation\
Основные результаты работы докладывались~на:
перечисление основных конференций, симпозиумов и~т.\:п.

\contribution\ Автор принимал активное участие \ldots

%\publications\ Основные результаты по теме диссертации изложены в ХХ печатных изданиях~\cite{Sokolov,Gaidaenko,Lermontov,Management},
%Х из которых изданы в журналах, рекомендованных ВАК~\cite{Sokolov,Gaidaenko}, 
%ХХ --- в тезисах докладов~\cite{Lermontov,Management}.

\ifthenelse{\equal{\thebibliosel}{0}}{% Встроенная реализация с загрузкой файла через движок bibtex8
    \publications\ Основные результаты по теме диссертации изложены в XX печатных изданиях, 
    X из которых изданы в журналах, рекомендованных ВАК, 
    X "--- в тезисах докладов.%
}{% Реализация пакетом biblatex через движок biber
%Сделана отдельная секция, чтобы не отображались в списке цитированных материалов
    \begin{refsection}%
        \printbibliography[heading=countauthornotvak, env=countauthornotvak, keyword=biblioauthornotvak, section=1]%
        \printbibliography[heading=countauthorvak, env=countauthorvak, keyword=biblioauthorvak, section=1]%
        \printbibliography[heading=countauthorconf, env=countauthorconf, keyword=biblioauthorconf, section=1]%
        \printbibliography[heading=countauthor, env=countauthor, keyword=biblioauthor, section=1]%
        \publications\ Основные результаты по теме диссертации изложены в \arabic{citeauthor} печатных изданиях\nocite{bib1,bib2}, 
        \arabic{citeauthorvak} из которых изданы в журналах, рекомендованных ВАК\nocite{vakbib1,vakbib2}, 
        \arabic{citeauthorconf} "--- в тезисах докладов\nocite{confbib1,confbib2}.
    \end{refsection}
}
При использовании пакета \verb!biblatex! для автоматического подсчёта
количества публикаций автора по теме диссертации, необходимо
их здесь перечислить с использованием команды \verb!\nocite!.
    


\chapter{Численные эксперименты и результаты} \label{chapt4}

\section{Методика верификации} \label{sect4_1}
Этот параграф про методику верификации прогнозов.

Пробная таблица методами TeXstudio
\begin{table}
    \centering
    \caption{Самостоятельно вставленная таблица на основании приведённой в щаблоне}%
    \label{tb:test}
    \renewcommand{\arraystretch}{1.5}
    \begin{tabular}{@{}@{\extracolsep{40pt}}llll@{}} %Вертикальные полосы не используются принципиально, как и лишние горизонтальные (допускается по ГОСТ 2.105 пункт 4.4.5) % @{} позволяет прижиматься к краям
    	\toprule     %%% верхняя линейка
    	Значения матриц	& ${\R}$ & ${\mP}$	& ${\mx}$	\\
    	\midrule %%% тонкий разделитель. Отделяет названия столбцов. Обязателен по ГОСТ 2.105 пункт 4.4.5 
    	Классический 	& ${x^{-1}}$ 	 & 8.77		& 8.77		\\
    	Ансамблевый		& 7.96 	 & 7.93		& 7.93		\\
    	Стохастический	& 8.72 	 & 8.77		& 8.77		\\
    	Детерминистский	& 8.72 	 & 8.77		& 8.77		\\
    	\bottomrule %%% нижняя линейка
    \end{tabular}%
\end{table}

\section{Используемые виды наблюдений} \label{sect4_2}

Описание боевой конфигурации касательно видов используемых наблюдений

\section{Подключение данных наблюдений AMV} \label{sect4_3}

Описание влияния подключения AMV на результаты усвоения

\subsection{Использование переопределения высоты наблюдений} \label{sect4_3_1}

Текст про использование переопределения высоты наблюдений и о влиянии этого приёма на результаты.

\subsection{Использование прореживания наблюдений} \label{subsect4_3_2}

Текст про использование переопределения высоты наблюдений с помощью коэффициента согласованности и о влиянии этого приёма на результаты.


\subsection{Использование недиагональной матрицы {$ \R $} } \label{subsect4_3_3}

Текст про использование недиагональной матрицы $ \R $ и про оказываемое влияние на результат. 

\subsection{Влияние данных AMV на результаты в приполярных регионах} \label{subsect4_3_4}

Усвоение AMV в приполярных регионах, получаемые результаты.
